\maketitle
\makesignature

\ifproject
\begin{abstractTH}

    เป้าหมายหลักของการทําโปรเจคนี้เกิดจาก \CI{ในปัจจุบันมีหลักสูตรที่คิดขึ้นและถูกใช้ในมหาวิทยาลัยอย่างหลากหลายโดยในแต่ละหลักสูตรนั้นก็จะมีโครงสร้างที่แตกต่างกันออกไป ถึงแม้ว่าจะเป็นหลักสูตรที่ถูกใช้ในคณะหรือสาขาเดียวกันแต่ถ้าเป็นคนละหลักสูตร ก็จะมีโครงสร้างของหลักสูตรที่แตกต่างกันอย่างแน่นอนเพราะเกิดการปรับปรุงทั้งเนื้อหาเเละโครงสร้างของหลักสูตรเพื่อความทันสมัยขององค์ความรู้}{ประโยคนี้ยาวมาก ต้องการสื่ออะไร} ยกตัวอย่างเช่น \CI{หลักสูตรปีการศึกษา 2558 และ หลักสูตรปีการศึกษา 2563 ของคณะวิศวกรรมศาสตร์ สาขาวิศวกรรมคอมพิวเตอร์ เมื่อนำเอาโครงสร้างของหลักสูตรมาเปรียบเทียบกันดูแล้ว จะพบว่ามีข้อแตกต่างกันในบางส่วน และมีความเหมือนกันในบางส่วนเช่นกัน}{ละเอียดเกินไปใน abstract แต่ก็ไม่ได้ดูจะลงรายละเอียดอะไรให้ชัดเจน} ซึ่งในแต่ละหลักสูตรก็จะมีความซับซ้อนของโครงสร้างหลักสูตรที่แตกต่างกันออกไปตามเกณฑ์ที่แต่ละคณะกำหนด ซึ่งในปัจจุบันเว็บไซต์ของ\CI{สำนัก-ทะเบียน}{ใช่เว็บไซต์นี้ที่แสดงโครงสร้างหลักสูตร?}นั้นมีความล้าสมัยในส่วนที่จะแสดงโครงสร้างของแต่ละหลักสูตร รวมไปถึง\CI{บางฟังก์ชันที่เกิดข้อผิดพลาด (bug)}{ฟังดูเป็นการกล่าวหาแบบลอยๆ ไร้หลักฐาน} และยัง\CI{ใช้เวลานานในการประมวลผล}{จะต้องประมวลผลอะไร} \CI{เช่น}{ที่ตามมาเป็นตัวอย่างของปัญหาไหน} การแสดงข้อมูลหลักสูตรรายบุคคลของนักศึกษาที่ยังไม่มีความละเอียดมากพอ และในการแก้ไขหลักสูตรในแต่ละครั้ง\CI{ของอาจารย์ผู้สอน}{ผู้สอนเป็นคนแก้หลักสูตร?}นั้นมีความยากลำบาก เช่น การเพิ่มวิชาเลือกเข้าไปในหลักสูตรทุกหลักสูตรของสาขาวิศวกรรมคอมพิวเตอร์ \CI{จำเป็นที่จะต้องเพิ่มทีละวิชาในทุกหลักสูตรที่ภาควิชามี}{จริงหรือ?} ทำให้เสียเวลานานพอสมควร ดังนั้นโครงงานนี้จึงมุ่งที่จะแก้ไขปัญหาและอุปสรรค\CI{ดังกล่าว}{สรุปจะแก้ทุกปัญหาที่กล่าวมาเลย? มันมีหลายปัญหามาก} โดยการสร้างโปรแกรมประยุกต์บนเว็บ (web application) สําหรับจัดการข้อมูลและโครงสร้างในแต่ละหลักสูตรให้มีความเข้าใจง่าย และสะดวกต่อการแก้ไข เพื่อเพิ่ม\CI{ประสิทธิภาพ}{ประสิทธิภาพของระบบนี้คืออะไร จะวัดยังไง}ในการใช้งาน เเละ\CI{ยั่งยืนในอนาคต}{ไม่เคยชอบวลีนี้ กว้างไป จะสื่ออะไรก็สื่อให้ชัดเจน} โดยมีความสามารถที่จะรองรับหลักสูตรในมหาวิทยาลัย และยืดหยุ่นต่อการใช้เรียกงานต่างๆ รวมไปถึงเพิ่มส่วนที่จะช่วยเเสดงข้อมูลการศึกษาของนักศึกษาอย่างละเอียด \CI{ให้กับอาจารย์ที่ปรึกษา และตัวนักศึกษาเอง}{เพิ่มแล้วมีประโยชน์อย่างไร} อาทิเช่น การแสดงรายวิชาที่ยังไม่ได้ทำการลงทะเบียน และการคำนวณเกรดล่วงหน้า เป็นต้น 

\vfill
\CIreply{เท่าที่เขียนมา ยังจับประเด็นไม่ค่อยได้ว่าปัญหามีอะไรบ้าง และเรากำลังจะพยายามทำอะไรบ้าง ดูเหมือนกำลังโฟกัสกับปัญหาที่ไม่สำคัญอยู่ด้วย อาจจะใช้เวอร์ชันด้านล่างนี้เป็นแนวทางในการแก้ไข}
{\color{Green4}
หนึ่งในเงื่อนไขเพื่อสำเร็จการศึกษาระดับปริญญาตรี คือ นักศึกษาจะต้องผ่านกระบวนวิชาในแต่ละกลุ่มที่กำหนดไว้ในหลักสูตร และต้องมีเกรดเฉลี่ยสะสมสำหรับวิชาเอกและสำหรับทุกวิชาไม่ต่ำกว่าขั้นต่ำ
\enskip ปัจจุบัน อาจารย์ที่ปรึกษามีหน้าที่ตรวจสอบการคาดว่าจะสำเร็จการศึกษา โดยพิจารณากระบวนวิชาต่างๆ ที่นักศึกษาได้ลงทะเบียนผ่านมาด้วยมือ ซึ่งก่อให้เกิดความผิดพลาดได้ง่าย
\enskip นอกจากนี้ นักศึกษาเองไม่มีเครื่องมือเพื่อตรวจสอบความคืบหน้าในการศึกษา ที่จะช่วยวางแผนให้จบการศึกษาภายในเวลาที่เหมาะสม
\enskip โครงงานนี้มีเป้าหมายที่จะพัฒนาเว็บแอพพลิเคชันสำหรับจัดเก็บและแสดงผลโครงสร้างหลักสูตรต่างๆ ที่หลากหลายและซับซ้อนในมหาวิทยาลัยเชียงใหม่ เพื่อเป็นแนวทางให้อาจารย์ที่ปรึกษาและนักศึกษาตรวจสอบความคืบหน้าในการสำเร็จการศึกษา รวมทั้งเพิ่มขีดความสามารถในการเพิ่มและปรับปรุงแก้ไขหลักสูตรต่างๆ ให้สะดวกรวดเร็วยิ่งขึ้นสำหรับผู้รับผิดชอบหลักสูตร
\enskip ในส่วนของการตรวจสอบความคืบหน้า ระบบจะแสดงรายการวิชาที่ยังจำเป็นต้องลงทะเบียน และมีตัวเลือกสำหรับคำนวณผลการเรียนล่วงหน้า
\enskip ในส่วนของการจัดการหลักสูตร ระบบจะมีส่วนต่อประสานผู้ใช้ที่เป็นมิตรและใช้งานง่าย เพื่อให้การแก้ไขหลักสูตรนั้นเป็นไปอย่างถูกต้องและเข้าใจตรงกันทุกฝ่าย
}
\end{abstractTH}

\begin{abstract}
The abstract would be placed here. It usually does not exceed 350 words
long (not counting the heading), and must not take up more than one (1) page
(even if fewer than 350 words long).

Make sure your abstract sits inside the \texttt{abstract} environment.
\end{abstract}

\iffalse
\begin{dedication}
This document is dedicated to all Chiang Mai University students.

Dedication page is optional.
\end{dedication}
\fi % \iffalse

\begin{acknowledgments}
Your acknowledgments go here. Make sure it sits inside the
\texttt{acknowledgment} environment.

\acksign{2020}{5}{25}
\end{acknowledgments}%
\fi % \ifproject

\contentspage

\ifproject
\figurelistpage

\tablelistpage
\fi % \ifproject

% \abbrlist % this page is optional

% \symlist % this page is optional

% \preface % this section is optional
