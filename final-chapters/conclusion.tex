\chapter{\ifenglish Conclusions and Discussions\else บทสรุปและข้อเสนอแนะ\fi}

\section{\ifenglish Conclusions\else สรุปผล\fi}

โครงงานระบบแสดงความคืบหน้าในการสำเร็จการศึกษา(Visualization system for graduation requirement fulfillment) 
จัดทำขึ้นเพื่อแก้ไขปัญหาความยุ่งยากในการตรวจสอบความคืบหน้าในการศึกษาของนักศึกษาจากวิธีเดิมที่จะต้องไล่ดูในหมวดวิชาต่างๆว่าลงเรียนวิชาในหมวดครบแล้วหรือยัง 
และช่วยให้นักศึกษาทำความเข้าใจหลักสูตรของตนเองได้ง่ายและเห็นภาพรวมของหลักสูตรเพื่อสามารถนำไปจัดการวางแผนการเรียนของตนเองได้ง่ายยิ่งขึ้น 
นอกจากนี้อาจารย์ที่ปรึกษาสามารถตรวจสอบความคืบหน้าในการศึกษาของนักศึกษาที่ตนเองดูแลอยู่ได้รวดเร็วและง่ายยิ่งขึ้น

\section{\ifenglish Challenges\else ปัญหาที่พบและแนวทางการแก้ไข\fi}

\subsection{Frontend}
\begin{itemize}
    \item  หน้าเว็บแอปพลิเคชันสำหรับอาจารย์ที่ปรึกษายังไม่ตอบโจทย์ต่อการใช้งานเท่าไรนัก 
    ควรเพิ่มเนื้อหาส่วนที่เป็นรายละเอียดของนักศึกษาเข้าไปด้วย เช่น tree-view 
    ของนักศึกษาคนแต่ละคนเพี่อให้ง่ายต่อการตรวจสอบความคืบหน้าของอาจารย์ที่ปรึกษา
    \item การ deploy มีข้อผิดพลาด คือหลังจากที่ผู้ใช้งาน login เข้าไปยังระบบหากเกิดข้อผิดพลาดในการดึงข้อมูลหน้าเว็บจะค้างที่หน้าสีขาว 
    ทำให้ผู้ใช้เกิดความสับสนและเสียเวลาในการรอหน้าเว็บแสดงข้อมูล ดังนั้นควรที่มีข้อความบอกถึงข้อผิดพลาดและสถานการณ์ดังกล่าวเพื่อแจ้งให้ผู้ใช้งานได้ทราบ
	
\end{itemize}

\subsection{Backend}
\begin{itemize}
    \item ในการประมวลผลเพื่อส่งข้อมูลไปยังหน้า student จําเป็นต้องใช้ หลาย query ทําให้ Frontend จําเป็นต้อง chain query หลายอันต่อเนื่องกัน ทําให้ระบบมีความช้าลง
	
\end{itemize}

\subsection{ปัญหาภาพรวม}
\begin{itemize}
    \item  หน้าเว็บแอปพลิเคชันสำหรับอาจารย์ที่ปรึกษายังไม่ตอบโจทย์ต่อการใช้งานเท่าไรนัก 
    ควรเพิ่มเนื้อหาส่วนที่เป็นรายละเอียดของนักศึกษาเข้าไปด้วย เช่น tree-view 
    ของนักศึกษาคนแต่ละคนเพี่อให้ง่ายต่อการตรวจสอบความคืบหน้าของอาจารย์ที่ปรึกษา
    \item หลังจากที่ผู้ใช้งาน login เข้าไปยังระบบหากเกิดข้อผิดพลาดในการดึงข้อมูลหน้าเว็บจะค้างที่หน้าสีขาว 
    ทำให้ผู้ใช้เกิดความสับสนและเสียเวลาในการรอหน้าเว็บแสดงข้อมูล ดังนั้นควรที่มีข้อความบอกถึงข้อผิดพลาดและสถานการณ์ดังกล่าวเพื่อแจ้งให้ผู้ใช้งานได้ทราบ
	
\end{itemize}


\section{\ifenglish%
Suggestions and further improvements
\else%
ข้อเสนอแนะและแนวทางการพัฒนาต่อ
\fi
}
\subsection{Frontend}
\begin{itemize}
    \item  หน้าเว็บแอปพลิเคชันสำหรับอาจารย์ที่ปรึกษายังไม่ตอบโจทย์ต่อการใช้งานเท่าไรนัก 
    ควรเพิ่มเนื้อหาส่วนที่เป็นรายละเอียดของนักศึกษาเข้าไปด้วย เช่น tree-view 
    ของนักศึกษาคนแต่ละคนเพี่อให้ง่ายต่อการตรวจสอบความคืบหน้าของอาจารย์ที่ปรึกษา
    \item หลังจากที่ผู้ใช้งาน login เข้าไปยังระบบหากเกิดข้อผิดพลาดในการดึงข้อมูลหน้าเว็บจะค้างที่หน้าสีขาว 
    ทำให้ผู้ใช้เกิดความสับสนและเสียเวลาในการรอหน้าเว็บแสดงข้อมูล ดังนั้นควรที่มีข้อความบอกถึงข้อผิดพลาดและสถานการณ์ดังกล่าวเพื่อแจ้งให้ผู้ใช้งานได้ทราบ
	
\end{itemize}

\subsection{Backend}
\begin{itemize}
    \item ระบบยังมีการประมวลผลที่ช้า เนื่องจากต้องเช็คข้อมูลใหม่ทุกครั้งที่มีการเรียก ควรเเก้ไขเป็นการเช็คเเค่ครั้งเดียวเเล้วเก็บ ข้อมูลทั้งหมดของนักศึกษาไว้
    \item ระบบยังไม่สามารถใช้กับ หลักสูตรระดับยากได้ เเต่การออกเเบบนั้นมีเเนวทางที่สามารถรองรับได้เเล้ว เเต่ยังขาดการทดลองจริง เพื่อหาข้อบกพร่องในการพัฒนาต่อ
	
\end{itemize}


